%% BioMed_Central_Tex_Template_v1.06
%%                                      %
%  bmc_article.tex            ver: 1.06 %
%                                       %

%%IMPORTANT: do not delete the first line of this template
%%It must be present to enable the BMC Submission system to 
%%recognise this template!!

%%%%%%%%%%%%%%%%%%%%%%%%%%%%%%%%%%%%%%%%%
%%                                     %%
%%  LaTeX template for BioMed Central  %%
%%     journal article submissions     %%
%%                                     %%
%%         <14 August 2007>            %%
%%                                     %%
%%                                     %%
%% Uses:                               %%
%% cite.sty, url.sty, bmc_article.cls  %%
%% ifthen.sty. multicol.sty		   %%
%%				      	   %%
%%                                     %%
%%%%%%%%%%%%%%%%%%%%%%%%%%%%%%%%%%%%%%%%%


%%%%%%%%%%%%%%%%%%%%%%%%%%%%%%%%%%%%%%%%%%%%%%%%%%%%%%%%%%%%%%%%%%%%%
%%                                                                 %%	
%% For instructions on how to fill out this Tex template           %%
%% document please refer to Readme.pdf and the instructions for    %%
%% authors page on the biomed central website                      %%
%% http://www.biomedcentral.com/info/authors/                      %%
%%                                                                 %%
%% Please do not use \input{...} to include other tex files.       %%
%% Submit your LaTeX manuscript as one .tex document.              %%
%%                                                                 %%
%% All additional figures and files should be attached             %%
%% separately and not embedded in the \TeX\ document itself.       %%
%%                                                                 %%
%% BioMed Central currently use the MikTex distribution of         %%
%% TeX for Windows) of TeX and LaTeX.  This is available from      %%
%% http://www.miktex.org                                           %%
%%                                                                 %%
%%%%%%%%%%%%%%%%%%%%%%%%%%%%%%%%%%%%%%%%%%%%%%%%%%%%%%%%%%%%%%%%%%%%%


\NeedsTeXFormat{LaTeX2e}[1995/12/01]
\documentclass[10pt]{bmc_article}    



% Load packages
\usepackage{cite} % Make references as [1-4], not [1,2,3,4]
\usepackage{url}  % Formatting web addresses  
\usepackage{ifthen}  % Conditional 
\usepackage{multicol}   %Columns
\usepackage[utf8]{inputenc} %unicode support
%\usepackage[applemac]{inputenc} %applemac support if unicode package fails
%\usepackage[latin1]{inputenc} %UNIX support if unicode package fails
\urlstyle{rm}
\usepackage{nameref}
 
 
%%%%%%%%%%%%%%%%%%%%%%%%%%%%%%%%%%%%%%%%%%%%%%%%%	
%%                                             %%
%%  If you wish to display your graphics for   %%
%%  your own use using includegraphic or       %%
%%  includegraphics, then comment out the      %%
%%  following two lines of code.               %%   
%%  NB: These line *must* be included when     %%
%%  submitting to BMC.                         %% 
%%  All figure files must be submitted as      %%
%%  separate graphics through the BMC          %%
%%  submission process, not included in the    %% 
%%  submitted article.                         %% 
%%                                             %%
%%%%%%%%%%%%%%%%%%%%%%%%%%%%%%%%%%%%%%%%%%%%%%%%%                     


\def\includegraphic{}
\def\includegraphics{}



\setlength{\topmargin}{0.0cm}
\setlength{\textheight}{21.5cm}
\setlength{\oddsidemargin}{0cm} 
\setlength{\textwidth}{16.5cm}
\setlength{\columnsep}{0.6cm}

\newboolean{publ}

%%%%%%%%%%%%%%%%%%%%%%%%%%%%%%%%%%%%%%%%%%%%%%%%%%
%%                                              %%
%% You may change the following style settings  %%
%% Should you wish to format your article       %%
%% in a publication style for printing out and  %%
%% sharing with colleagues, but ensure that     %%
%% before submitting to BMC that the style is   %%
%% returned to the Review style setting.        %%
%%                                              %%
%%%%%%%%%%%%%%%%%%%%%%%%%%%%%%%%%%%%%%%%%%%%%%%%%%
 

%Review style settings
%\newenvironment{bmcformat}{\begin{raggedright}\baselineskip20pt\sloppy\setboolean{publ}{false}}{\end{raggedright}\baselineskip20pt\sloppy}

%Publication style settings
%\newenvironment{bmcformat}{\fussy\setboolean{publ}{true}}{\fussy}

%New style setting
\newenvironment{bmcformat}{\baselineskip20pt\sloppy\setboolean{publ}{false}}{\baselineskip20pt\sloppy}

% Begin ...
\begin{document}
\begin{bmcformat}


%%%%%%%%%%%%%%%%%%%%%%%%%%%%%%%%%%%%%%%%%%%%%%
%%                                          %%
%% Enter the title of your article here     %%
%%                                          %%
%%%%%%%%%%%%%%%%%%%%%%%%%%%%%%%%%%%%%%%%%%%%%%

% TODO: come up with a better title, or at least a subtitle?

\title{Structure-based classification and ontology in chemistry}
 
%%%%%%%%%%%%%%%%%%%%%%%%%%%%%%%%%%%%%%%%%%%%%%
%%                                          %%
%% Enter the authors here                   %%
%%                                          %%
%% Ensure \and is entered between all but   %%
%% the last two authors. This will be       %%
%% replaced by a comma in the final article %%
%%                                          %%
%% Ensure there are no trailing spaces at   %% 
%% the ends of the lines                    %%     	
%%                                          %%
%%%%%%%%%%%%%%%%%%%%%%%%%%%%%%%%%%%%%%%%%%%%%%


\author{Janna Hastings\correspondingauthor$^1,^2$%
         \email{Janna Hastings\correspondingauthor - hastings@ebi.ac.uk}
       \and 
         Despoina Magka$^3$%
         \email{despoina.magka@cs.ox.ac.uk}%  logics
       \and 
         Lian Duan$^1$%
         \email{dlian@ebi.ac.uk} %  chemoinformatics
       \and 
         Robert Stevens$^4$  % chemistry, ontology technology
         \email{robert.stevens@manchester.ac.uk}%
       \and 
         Marcus Ennis$^1$ % chebi
         \email{mennis@ebi.ac.uk}        
       and 
       	 Christoph Steinbeck$^1$%
         \email{steinbeck@ebi.ac.uk}%
      }
      

%%%%%%%%%%%%%%%%%%%%%%%%%%%%%%%%%%%%%%%%%%%%%%
%%                                          %%
%% Enter the authors' addresses here        %%
%%                                          %%
%%%%%%%%%%%%%%%%%%%%%%%%%%%%%%%%%%%%%%%%%%%%%%

\address{%
    \iid(1)Chemoinformatics and Metabolism, European Bioinformatics Institute, Hinxton, UK\\
    \iid(2)Swiss Center for Affective Sciences, University of Geneva, Switzerland\\
    \iid(3)Department of Computer Science, University of Oxford, UK\\
    \iid(4)Manchester University, UK%\\
}%

\maketitle

%%%%%%%%%%%%%%%%%%%%%%%%%%%%%%%%%%%%%%%%%%%%%%
%%                                          %%
%% The Abstract begins here                 %%
%%                                          %%  
%% Please refer to the Instructions for     %%
%% authors on http://www.biomedcentral.com  %%
%% and include the section headings         %%
%% accordingly for your article type.       %%   
%%                                          %%
%%%%%%%%%%%%%%%%%%%%%%%%%%%%%%%%%%%%%%%%%%%%%%


\begin{abstract}
        % Do not use inserted blank lines (ie \\) until main body of text.
Recent years have seen an explosion in the availability of data in the chemistry domain. With this information explosion, however, retrieving \textit{relevant} results from the deluge of available information, and \textit{organising} those results, become ever harder problems. Computational processing is essential to filter and organise the available resources so as to better facilitate the work of scientists. Ontologies encode expert domain knowledge in a hierarchically organised computable format. One such ontology for the chemical domain is ChEBI. ChEBI provides a classification based on structural features and a role or activity-based classification. An example of a structure-based class is `pentacyclic compound' (compounds containing five-ring structures), while an example of a role-based class is `analgesic', since many different chemicals can act as analgesics without sharing structural features. 

Structure-based classification in chemistry exploits elegant regularities and symmetries in the underlying chemical domain.  Some of these regularities lead easily to automated approaches to classification, and indeed hierarchical organisation through clustering or scaffolding are well developed in chemoinformatics.  Logical definition of chemical classes in ontologies allows automated reasoning to detect class membership and manage hierarchies automatically.  Some relevant and interesting chemical classes cannot yet adequately be automated.  In this paper, we describe several categories of structural classification in chemistry.  We compare these patterns of class definition to tools which allow for automation of hierarchy construction within chemoinformatics and within chemical ontology.  Finally we discuss the relationships between chemoinformatics approaches and logic-based approaches. 

Systems which perform intelligent reasoning tasks need to rely on a diverse set of underlying computational utilities including algorithmic, statistical and logic-based tools.  For the task of automatic structure-based classification of chemical entities, essential to managing the vast swathes of chemical data being brought online, systems which are capable of hybrid reasoning combining several different approaches are crucial.  

\end{abstract}



\ifthenelse{\boolean{publ}}{\begin{multicols}{2}}{}




%%%%%%%%%%%%%%%%%%%%%%%%%%%%%%%%%%%%%%%%%%%%%%
%%                                          %%
%% The Main Body begins here                %%
%%                                          %%
%% Please refer to the instructions for     %%
%% authors on:                              %%
%% http://www.biomedcentral.com/info/authors%%
%% and include the section headings         %%
%% accordingly for your article type.       %% 
%%                                          %%
%% See the Results and Discussion section   %%
%% for details on how to create sub-sections%%
%%                                          %%
%% use \cite{...} to cite references        %%
%%  \cite{koon} and                         %%
%%  \cite{oreg,khar,zvai,xjon,schn,pond}    %%
%%  \nocite{smith,marg,hunn,advi,koha,mouse}%%
%%                                          %%
%%%%%%%%%%%%%%%%%%%%%%%%%%%%%%%%%%%%%%%%%%%%%%




%%%%%%%%%%%%%%%%
%% Background %%
%%
\section*{Background}

Recent years have seen an explosion in the availability of data throughout the natural sciences. Primary data facilitates research through complex data-mining and knowledge discovery methods. However, with this information explosion, retrieving \textit{relevant} information from the deluge has become much more difficult. Computational processing is essential to filter, retrieve and organise such data. These goals are facilitated by \textit{formal ontologies}: machine-understandable\footnote{replaced computable with machine-understandable} encodings of human domain knowledge. These ontologies are used in several different ways \cite{lambrix2004}: to provide standardisation of terminology and identification across all entities in a domain so that multiple sources of data can be aggregated through comparable reference terms, to provide hierarchical organisation so that such aggregation can be performed at different levels and data can be browsed in an easily accessible fashion, and to allow for logic-based intelligent applications to be built which are able to perform complex reasoning tasks such as checking for errors and inconsistencies and deriving logical inferences. Logic-based knowledge representation (where ontologies serve as knowledge engineering artefacts)\footnote{changed content of parenthesis} can be contrasted with algorithmic ``knowledge" representation, in which software algorithms procedurally define outputs based on stated inputs, and with statistical ``knowledge" representation, in which complex statistical models are trained to produce outputs based on a given set of inputs by learning weights for a complex set of internal parameters.  The advantages of logic-based knowledge representation is that it allows the knowledge to be expressed \textit{as knowledge}, i.e. as statements which are easily comprehensible, true and self-contained.  Statistical methods operate as black boxes and procedural methods require a programmer in order to manipulate or extend them. 

One example of a successful ontology in the biomedical domain is the Gene Ontology \cite{go2000}, which is used \textit{inter alia} for the statistical analysis of large-scale genetic data to identify genes which are significantly enriched for specific functions.  For the domain of biologically interesting chemistry, the Chemical Entities of Biological Interest ontology (ChEBI) \cite{chebi2010} provides a classification of chemical entities such as atoms, molecules and ions.  ChEBI organises chemical entities according to shared structural features, for example, carboxylic acids are all molecular entities that possess the characteristic carboxy group, and according to their activities in biological and chemical contexts, for example, acting as an antiviral agent. ChEBI is widely used as a database of chemical entities which can be queried both by structural classes and by functional annotations in the role ontology. The ontology has been applied in diverse applications such as annotation of chemicals in biological databases for pathways, interactions, and systems biology models \cite{matthews2009,libiomodels2010,kerrien2007}; chemical text mining \cite{corbett2006}; formalising the chemistry underlying biological ontologies \cite{mungall2010}; semantic similarity \cite{couto2010}; and metabolome prediction \cite{swainston2010}. Like the GO, ChEBI is manually maintained by a team of expert curators.  Historically, bio-ontologies such as GO and ChEBI have been developed as Directed Acyclic Graphs (DAGs), a deliberately simplified ontology format which allowed domain experts (non-logicians) to directly participate in ontology engineering. However, with the increasing availability of supporting tools and widespread adoption, there is a growing trend of evolution  towards the Web standard ontology language OWL \cite{OWL2NextStep}\footnote{changed the url to the (standard) owl 2 journal publication}, which provides a sophisticated suite of logic-based constructs to support eloquent knowledge representation and automated reasoning in real-world domains \cite{alterovitz2010}.

Structure-based classification in chemistry exploits elegant regularities and symmetries in the underlying chemical domain.  Some of these regularities lead easily to algorithmic or statistical automated approaches to classification, and as a result, algorithmic approaches to hierarchical organisation of large-scale compound collections is a mature research area in chemoinformatics \cite{barnard1992,deshpande2005}. However, there are nevertheless several key benefits to adoption of the ontology-based approach in the chemistry domain, namely:
\begin{itemize}
	\item Classification knowledge represented in an ontology is \textit{explicit}, while algorithms which perform hierarchical classification often act as \textit{black boxes}, and to change the classification requires changes to the underlying software or re-training a complex statistical model;
	\item Using an ontology for classification allows for \textit{explanations} (justifications) \cite{horridgeentail09}, both for computed classifications and for computed inconsistencies. This can be contrasted to black-box approaches such as neural networks which cannot offer any explanations. 
	\item Representation of chemical knowledge in an ontology allows it to be harnessed in a generic fashion from within diverse knowledge-based applications which also utilize knowledge from \textit{other domains} (a core requirement for whole-scale systems biology), while to make use of chemoinformatics algorithms and toolkits requires custom software, differing from the software used in other domains; and
	\item There are several types of chemical class definition which are not adequately represented in algorithmic approaches, which can be formalised in logical expressions (although not always in straightforward OWL). 
\end{itemize}

Conversely, there are several benefits to adopting chemoinformatics tools within the ontology engineering process in the domain of chemistry, such as to benefit from the well-developed and rapid algorithms for detecting parthood between chemicals and computing properties.  In this paper, we present an analysis of the types of structure-based classes used in chemistry, and then discuss these as compared to available chemoinformatics algorithms for computing hierarchical ordering on sets of compounds and to the available constructions for encoding class definitions in chemical ontologies based on OWL.  We then explore aspects of chemoinformatics tools which we believe can enhance chemical ontology and explore problem areas in common to both fields which we believe can provide a stimulus for future research bridging the domains of chemoinformatics and computational logic. 

The remainder of this paper is organised as follows. The remainder of this Background presents some relevant chemistry and ontology concepts.  Following that, in our Results we firstly present the types of classes used in chemical classification and thereafter compare these types of classes to the capabilities of chemoinformatics hierarchy construction methods in chemoinformatics and that of chemical ontologies. In our Discussion, we discuss the relationship between chemoinformatics and logical approaches and some applications of chemical ontology.  We conclude with our outlook and open research areas. 



\subsection*{Classification in chemistry}
\label{sec:backclassif}

The ability to \textit{classify} raw information into meaningful groups is an essential component of human intelligence, which thus far has proven difficult to replicate in machine reasoning.  In particular, classification has a long tradition in chemistry:  the periodic table of the elements is one of the longest-standing and most used systems of classification throughout the natural sciences.  

The benefits of classification systems are severalfold. Classification \textit{organises} large volumes of information into sensible groupings so that they are more accessible to humans. Such hierarchical organisations can be more easily browsed; research in cognitive science shows that humans can only browse and compare a relatively low number of concepts at the same level at the same time, thus grouping into hierarchies reduces the amount of detail that has to be dealt with at each level \cite{sternberg2003}. A hierarchical structure allows narrowing in on the area of interest within a large domain, and only exploring the details of that narrowed in area, rather than observing the full domain at such a detailed level. 

A second benefit of a hierarchical organisation is that it allows for the compact representation of generalised knowledge at the highest level to which it applies.  For example, statements which are true for all mammals need to be expressed at the level of mammals as a whole, and not repeated for every specific mammal that occurs downstream.  Similarly, features which apply to all carboxylic acids can be expressed at the level of carboxylic acids as a whole, rather than repeated at the level of the different molecules as is required in databases or other flat structures which allow no general grouping or hierarchical organisation. 

Hierarchical organisation of knowledge in a domain allows for data-driven discovery, enabling useful predictions to be made.  For example, in functional genomics, the analysis of large-scale genetic data is facilitated by the grouping together of different genes which perform the same function, and modular analysis of such datasets reveals organisation at an aggregate level which is sometimes not apparent at the level of the raw data due to overloading of detail and noise in the underlying signal. Hierarchical organisation of knowledge also allows useful predictions to be made, since it allows generalisation of knowledge to the highest possible level of applicability, and consequent prediction of properties of novel discovered members of the class.  

Chemical classes, the objects found within a chemical classification system, group together chemical entities in meaningful, scientifically relevant hierarchy. Ideally, all members of a chemical class should share important dispositional properties such as chemical reactivity. In fact, almost the only methods of classification available to historical chemists, before compound structures were well understood, were (i) based on the observation of reactivity through means of performing controlled reactions between different substances; or (ii) based on the origin of the molecule, when the molecule was isolated from a natural product substance. Much of these historical forms of classification are still inherited today and are taught in chemistry classes and reproduced in textbooks. Knowledge about the structural features which form the underlying causes of the shared dispositional properties (where such existed), and the structural features shared between similar natural product substances, was only developed later. However, now that chemical structures are well described (within the limits of the chemical graph formalism), many more structural features are able to be used for chemical class definitions. 

Interesting classes in chemistry can be grouped into those which are structure-based and those which are not. Structure-based classes are defined based on the presence of some shared structural feature across all members of the class.  This feature, however, may be crisply defined or vaguely defined.  Crisply defined structural classes will form the bulk of our focus in this paper, and are discussed further in the section \textit{\nameref{sec:resultsclasses}} below.   

Vaguely defined structural classes are those that are based on a family resemblance between a group of molecules, which are often of natural origin or have biological relevance.  For example, \textit{steroids} are defined as ``Any of naturally occurring compounds and synthetic analogues, based on the \textit{cyclopenta[a]phenanthrene carbon skeleton}, partially or completely hydrogenated; there are usually \textit{methyl groups} at C-10 and C-13, and often an \textit{alkyl group} at C-17. By extension, one or more bond scissions, ring expansions and/or ring contractions of the skeleton may have occurred."  The vagueness is indicated by terms and phrases such as `usually'; `one or more\ldots may have'.

Chemical classes can also be defined based on where the chemical came from in synthetic or natural pathways. Chemicals of natural metabolic origin are called natural products.  For example, alkaloids are defined as ``Any of basic nitrogen compounds (mostly heterocyclic) occurring mostly in the plant kingdom (but not excluding those of animal origin). Amino acids, peptides, proteins, nucleotides, nucleic acids, amino sugars and antibiotics are not normally regarded as alkaloids. By extension, certain neutral compounds biogenetically related to basic alkaloids are included."

Many interesting classes of chemicals are defined based on what the chemical does (its function or activity) in a biological or chemical context. Included in this group are drug usage classes such as antidepressant and antifungal; chemical reactivity classes such as solvent, acid and base; and biological activities such as hormone \cite{batchelor2010}. 

%Re: certain types of uncertainty in class definitions in terms of contextual information, perhaps can cite Bio-ontologies and OWLED papers, in the Discussion where the need for hybrid systems for knowledge representation are discussed. 

Hybrid classes:  tricyclic antidepressant; tetracyclic antibiotic and so on \ldots  Other types of non-structural or vague classes? 


% This is a background section. 
\subsection*{Logic-based reasoning and ontology}
\label{sec:backlogic}

Logic-based representation lies at the heart of modern knowledge representation (KR) technologies. It employs formal methods developed in the context of mathematical logic in order to encode knowledge about the world. The key requirement for these methods is that the knowledge is stored in a machine-processable format. All though a whole range of KR formalisms have been investigated, a core feature that KR languages share is the use of a well-defined syntax and semantics. The syntax serves as the alphabet of the language: it provides a set of symbols and a set of rules that regulate the arrangement of the symbols. The semantics enriches the syntactic objects with a meaning so that syntactic expressions (also known as \emph{axioms}) have a universal and predefined interpretation. A set of axioms that model domain-specific knowledge about an aspect of the world constitutes an \emph{ontology}. 

The amenability of KR languages to automated reasoning is of crucial importance. A reasoning algorithm --relying on principles of logical deduction-- computes the inferences that follow from a  set of formally defined axioms; note that a reasoning algorithm is uniquely associated with a KR language, that is it is tied to the specific syntax and semantics of the language. A reasoning engine can be used to check the logical consistency of a set of logical axioms. For instance, if a knowledge base (i) defines the organic and inorganic compounds as disjoint chemical classes (ii) contains the fact that cobalamin is an organic compound and (iii) also classifies cobalamin as inorganic, then a contradiction will be detected. Another standard reasoning task is the discovery of information that is not explicitly stated in the ontology. For example, if an ontology categorises cobalamin as B vitamin and also asserts that B vitamins participate to cell metabolism, then the fact that cobalamin participates to cell metabolism is derived. The automation of the above --traditionally performed by humans-- tasks has a clear advantage as it permits the allocation of research resources to more intellectually intense activities.

A reasoning procedure is required to exhibit certain properties in order to be practically useful. Namely, a reasoning algorithm needs to derive \emph{correct} inferences, that is inferences that are in accordance with the semantics of the language; this property is known as \emph{soundness}. Additionally a reasoning algorithm oughts to be \emph{complete}, i.e. to compute \emph{all} the correct inferences that are entailed by a set of axioms. Finally, an essential need for a reasoning algorithm is to \emph{terminate}, that is to halt after a finite amount of time. A vital contribution of logic is that it can offer guarantees --by means of formal proofs-- for the soundness, completeness and termination of a reasoning algorithm for \emph{all} input ontologies. A KR formalism for which a sound, complete and terminating reasoning algorithm exists is (informally) called \emph{decidable}.\footnote{According to the formal definitions of logic, the \emph{problem} of deciding whether a knowledge base is inconsistent is (un)decidable, rather than the actual language.} As a consequence, decidability is a highly desirable feature for a logic-based formalism in use.

Apart from decidability, another important feature of KR formalisms is tractability, that is how expensive the reasoning tasks are in terms of computational resources, e.g. performance time. The trade-off between the expressive power and the tractabiligy of a logic-based language  is a fundamental one: increasing the expressivity of the language usually results in a more resource-consuming reasoning algorithm or even undecidability. For instance, consider first-order logic (FOL) and propositional logic (PL): FOL allows to model a much broader range of situations than PL, as e.g. that for every molecule X, if X is organic and contains a hydroxyl group, then X is an alcohol. Nevertheless, reasoning in propositional logic is decidable, whereas reasoning tasks in unrestrained first-order logic are undecidable.

The need for decidable formalisms has been the driving force behind the development of Description Logics (DLs), a family of logic-based languages with well-understood computational properties that allow for different sorts of definitorial statements when building a knowledge base. DLs serve as the underlying formalism for the Web Ontology Language (OWL) which is a W3C-standardised  knowledge representation language that is widely used both in industry and in academia; ontologies such as GO or ChEBI are expressed in OWL. A powerful feature of OWL is the ability to perform \emph{automatic classification} based on \emph{full class definitions}. A full class definition is specified in terms of necessary and sufficient conditions for class membership. For instance, the OWL axiom $\mathsf{MetallicCompound} \equiv \mathsf{Compound \sqcap \exists hasAtom.Metal}$ states that an object is a metallic compound if and only if it is a compound and has a metal atom. Due to the machine-understandable nature of OWL definitions, OWL is the standard knowledge representation and reasoning language for use in the Semantic Web, which differentiates itself from the Syntactic Web by associating Web documents with a well-defined meaning. Significant advances in the DL research lead to the release of OWL 2\footnote{http://www.w3.org/TR/owl2-overview/} which improves on the first version of OWL by providing new expressive features, such as more advanced handling of data values and ranges (i.e. qualified cardinality restrictions and property chains).

In spite of the fact that OWL is an excellent KR formalism for the encoding of tree-like structures, it is fundamentally inable to correctly represent cyclic structures, such as molecular entities with rings \cite{magka2010}. OWL exhibits the tree-model property \footnote{VardiModalLogic} which on the one hand ensures important computational properties, such as decidability, but on the other hand prevents the users from describing  non-tree-like structures using OWL axioms. For instance, one may state using OWL axioms that cyclobutane has four carbon atoms, but it is not possible to specify that these four atoms are arranged in a ring. Therefore, one of the prevailing challenges in chemical knowledge representation is crafting logic-based formalisms that are able to faithfully represent cyclic structures and, thus, pave the way for ontology-based applications that aim to classify chemical compounds.

%  DESPOINA TO HELP WITH THIS SECTION (i have now added text to this section)
 
%% ANY OTHER LOGIC BACKGROUND NEEDED for the remainder of the paper? (so far no, but we will see whether we need any when we will be done with the classificaiton section)


\section*{Results}

\subsection*{Types of structurally defined chemical classes}
\label{sec:resultsclasses}

The chemical graph formalism \cite{trinajstic1992}allows the description of fully defined chemical classes at a certain level of granularity, for example, the class of all paracetamol molecules.  Paracetamol molecules all have in common the connectivity as described in the graph describing the class, but individual molecules vary in conformation and configuration depending on their environmental context. The chemical graph formalism, however, only applies to fully defined chemical structures, and there are limited extensions to support higher-level modelling (such as R groups). 

There are many different types of classes that are used in chemistry, for various reasons of historical development and usefulness. New classes are both created and discovered, as research in natural chemistry and biochemistry reveals new information, so too synthetic chemists are able to create ever more complex and creative structures.

Some of the elements which are used in chemical class definition include: 
\begin{enumerate}
	\item	Interesting parts
	\item	Basic chemical properties, such as charge
	\item	Atomic composition
	\item	Cycles and ring system arrangements
	\item	Properties based on entire molecular structures
	\item	Structural formulae
\end{enumerate}

Most of these elements can be used singly or in combination with other elements.  Further explanations as well as examples follow in the sections below. 


\subsubsection*{Interesting parts}

Perhaps the most prominent of methods for classifying chemical entities based on features of their structures is based on the presence or absence of specific parts.  Such parts may be the overall `skeleton' of the structure or they may be minor constituents. The skeleton is usually loosely defined as the major or most relevant part of the molecule, the `backbone' to which other groups are attached as decorations. 

For example, metalloporphyrins is defined as any compound containing a porphyrin skeleton and a metal atom.  Note that as the term is commonly used in chemistry, a skeleton is not always a straightforward substructure, since bond order (single or double) may vary, and bonds may even be added or removed while retaining the same skeleton. For example, \ldots
 
Allowing variance in bond order, or the addition or removal of parts of the skeleton, gives rise to a vague class definition. Here, therefore, we focus on the stronger sense of skeleton that implies that the skeleton as specified must be a substructure of the molecule it is a skeleton of. Classes defined with skeletons in this fashion are often named for the skeleton.  However, the same name is often used to mean a class of compounds with a certain skeleton and the larger class of compounds containing a part which has that skeleton (X and X derivatives). 

Parts may also be straightforward constituents in which there is no implication that the part is somehow maximal, as there is in the case of skeletons.  General parts are termed `groups' (which may be simple atoms). The number (count, cardinality) of such groups is also important.  For example, tricarboxylic acids can be defined as a compound containing exactly three carboxy groups. 

The class definition may also specify the position at which a group (or set of groups) is attached to a skeleton. Such positions are assigned by rules for numbering the skeleton of a molecule in a reproducible (and community-agreed-upon) fashion.  For example, \ldots

Some particularly problematic classes refer to the relative arrangement of parts / attachments within the whole molecule. A special case is the relative configuration of stereocentres. Chemical graphs can be specified for completely stereochemically specified entities, and for completely stereochemically unspecified entities, but relative configurations of stereogenic centers cannot be specified using traditional chemical graph representation formalisms. For example, `allothreonine' [rel-(2R,3R)-2-amino-3-hydroxybutanoic acid] and `threonine' [rel-(2R,3S)-2-amino-3-hydroxybutanoic acid] are compounds with a relative configuration of stereogenic centres, thus for which a graph cannot currently be drawn. What cannot be represented in the graph formalism is a relative arrangement of these - if one is up, the other is down - or if one is down, the other is down - for example. 


\subsubsection*{Basic chemical properties such as charge}

Straightforward chemical properties such as charge and unpaired electrons are used to define broad classes of molecules such as ion and radical. Aromaticity and saturation are other properties commonly found in class definitions. These also apply at a lower level of classification, such as `aromatic diazonium ion'.  


\subsubsection*{Atomic composition}

Classes may be defined based on the presence of certain types of atoms, organised according to the layout of the periodic table.  Examples are `carbon molecular entities' and `lanthanoid molecular entities'. Classes qualify as subclasses of carbon molecular entities if they contain any atom of carbon, regardless of what other atoms they contain in addition.  Classes qualify as subclasses of lanthanoid molecular entities if they contain any of the lanthanoid group atoms.  As most complex molecular entities belong to several such classes, automation of this aspect of classification is obviously highly desirable. 


\subsubsection*{Cycles and ring system arrangements}

Another element commonly used in class definitions is the number and arrangement of rings (cycles) in a ring system which is a part of the molecule. For example, the classes `ring assembly' and `polycyclic cage' both refer, in their definitions, to numbers and arrangements of rings in the molecule. Polycyclic cages are molecules which are composed entirely of cycles which are fused together in such a way as to form an overall cage-like structure.  Examples are the fullerenes, nanotubes, and small regular compounds such as cubane. 

Number and counts of rings: tetracyclic; pentacyclic; \ldots

ortho- and peri-fused compound: A polycyclic compound in which one ring contains two, and only two, atoms in common with each of two or more rings of a contiguous series of rings. Such compounds have n common faces and less than 2n common atoms.

ortho-fused compound: A polycyclic compound in which two rings have two, and only two, atoms in common. Such compounds have n common faces and 2n common atoms.

Related to the chemical properties from the previous section, the `cyclic' modifier is often treated as an overall property of the molecule and as a modifier for other class types. Consider: `cyclic ketone', `cyclic peptide', `cyclic ether', `cyclic tetrapyrrole', \ldots


\subsubsection*{Properties of entire molecular structures}

Chemicals defined based on shapes: Molecular M\"{o}bius strips and knots such as catenanes, rotaxanes and trefoil knots. 



\subsubsection*{Structural formulae}
\label{sec:molformula}

Another form of definition by atomic composition is the definition of classes of molecular entity based on specifying the \textit{exclusive} atomic composition.  This can be contrasted to parthood (where other attachments are allowed). 

Finally, an interesting, yet problematic to depict with existing graph-based tools, element of chemical class definitions, is that based on structural formula. 

For example, `alkane' is defined as ``an acyclic branched or unbranched hydrocarbon having the general formula $C_{n}H_{2n+2}$.'' 

Also, polymers (strings of repeating units).


\subsubsection*{Hybrid classes}

Given all of the above different types of elements of chemical class definitions, perhaps the most interesting classes arise in combinations of the various different elements, including combinations of structure-based class elements and non-structure-based groupings:  tricyclic antidepressant; tetracyclic antibiotic and so on...



%%
%%  Compare the above class types to the various available methods of automatic hierarchy construction in chemoinformatics
%%  Intended to also serve as a related work review on classification methods in chemoinformatics
%%
\subsection*{Algorithmic and statistical approaches to automatic hierarchy construction}
\label{sec:resultscheminf}

%% DUAN TO HELP WITH THIS SECTION

The interesting question is -- given the above types of chemical class definitions, and their implied hierarchical relationships, to what extent can the existing chemoinformatics approaches construct hierarchies in which the nodes correspond to classes defined in the above way? 

The basic units of chemoinformatics approaches are the substructure search, similarity search and graph isomorphism search.  Substructure search in particular can be used directly to address the construction of hierarchies by parts. 

SMARTS and R-GROUPS

Machine learning then has to be discussed.  In particular the way that machine learning allows application of certain types of hierarchy construction on a larger scale than exhaustive pairwise alignment?  

% SIMILARITY
Chemoinformatics solutions have been developed which facilitate classification of sets of chemical entities by automatic methods.  Such automatic classification systems are in common use in many areas of chemistry including drug discovery. One example of such a classification technique is classification based on \textit{hierarchical similarity clustering} \cite{barnard1992}.  In this technique, similarity between compounds is computed on a pairwise basis, and compounds are then grouped together in clusters of high mutual similarity. There are many different algorithms for computing similarity measures between compounds and for aggregating compounds into clusters based on pairwise similarity measures. Different algorithms lead to different classification results. 
%Limitations of this approach?  
One limitation is that most similarity measures, in order to be efficiently computable, compute only a subset of the total features of the molecules concerned, and local paths predominate over overall molecular structure. Thus, molecules may turn out to be quite similar according to such an algorithm while containing overall a rather different structure. Examples are xxxx and yyyy.  This problem is exacerbated for molecules of high structural regularity (e.g. polycyclic carbon compounds).  Nevertheless, similarity landscapes are of paramount importance in reducing the complexity and understanding the features of large collections of compounds. % Get Duan to check this section and the next. 

% SCAFFOLDS (PARTHOOD)
Another classification technique which is of particular importance in the pharmaceutical industry is classification based on substructures \cite{deshpande2005} or (relatedly) molecular scaffolding \cite{FindSomething}.  In this technique, substructures of the molecule are taken at increasing size, and molecules are classified based on the presence of such substructures.  Classification based on scaffolds form a hierarchy based on composition, and in pharmaceutical chemistry organic ring structures are usually taken as the backbone.  Scaffolds are then decorated with a variety of functional groups, and scaffolds may be combined to form scaffolds of increasing complexity. 
% This is DEFINITELY best illustrated with a diagram. 
Scaffolds often fix the overall structure of the molecule, which in bioactive and especially in synthetic chemistry has a large influence on the activity of the molecule in the biological system.  However, hierarchies based on scaffolding do not allow for the specification of overall properties of the molecule, and also does not allow for the annotation of similar aspects of molecules aside from their scaffolds. 
% Further limitations?

% The concept of scaffold may be as vague as skeleton;  it may be that X-containing entities are classified or instead that entities that are largely or mostly X are classified. If the latter, it would be very interesting to see how this is defined, as this can help us.  If the former, it is very similar to the approach which Leo and Michel have developed and reported on (cited later).


%%
%%  Compare the above class types to the various available methods of classification in chemical ontologies
%%  Intended to also serve as a related work review on chemical ontology
%%
\subsection*{Automatic classification in chemical ontologies}
\label{sec:resultschemontology}

%% DESPOINA TO HELP WITH THIS SECTION

Assuming a background in which full class definitions expressed in chemical ontologies enables automated reasoning both to check class membership and to construct appropriate hierarchies, to what extent do the logical facilities available to ontologies allow the construction of definitions according to the patterns outlined above? 

General comments about classification of chemicals in ontologies should go here, and then be specialised in the different relevant sections below addressing the specific types of 

This section could also be ordered differently: rather than ordering by chemical class type (requirement) it could be ordered by ontology technology / formalism. 

\subsubsection*{Atomic composition}

Robert Stevens describes an OWL-ification of the periodic table here: http://robertdavidstevens.wordpress.com/2011/05/05/an-ontology-of-the-periodic-table-using-electronic-structure-of-the-atom/



\subsubsection*{Interesting parts}

The representation of, and classification of, chemicals based on their interesting parts in the form of functional groups is probably the most well developed area of OWL-based chemical ontology at present. Dumontier et al. \cite{dumontier2007} developed an ontology of chemical functional groups encoded in OWL, which was able to be used to automatically link chemical entities defined by means of graphs to ontological classes defined by means of functional groups. They made use of a cheminformatics software application to extract functional groups from molecules with chemical structures specified in standard cheminformatics representation formalisms. While not as rich as the feature term representation described above, and far smaller than ChEBI ontology, the Dumontier representation allows powerful use to be made of the relatively compact knowledge base of functional groups in terms of the classification of arbitrary molecules in a defined hierarchy. Care, however, had to be taken to only encode the structures of the functional groups insofar as they were not cyclic. Along similar lines is the OWL encoding used by the Lipid Ontology \cite{chepelevlipids2011} for automatic classification of lipids. 


\subsubsection*{Cycles and ring arrangements}

What can currently be done to represent and reason over cycles in structured objects -- e.g. using DL safe rules? Advantages and limitations of various approaches. 

(Question that came up in presentation in Buffalo) To what extent can cardinality be used to create a basic cyclic model from OWL axioms?  e.g. stating that molecule has-part ONLY six atoms and that each atom is connected-to ONLY two other atoms, what models do you get? 
 

What is the ongoing research in the field of logical knowledge representation and reasoning that will allow such representation to a greater extent than before?  (Despoina's work)
 
An approach for the representation of the overall structure of highly regular polycyclic molecules is set out in \cite{hastings2011} using a combination of monadic second-order logic and ordinary OWL.  This approach has not yet been implemented in practice but shows promise for logical reasoning over features involving regularity in the overall structure of molecules, which features are not well able to be captured in general with parts-based approaches. 
  

\subsubsection*{Molecular formulae}

Closure axioms will work for `only'-type atomic composition (hydrocarbons); however, this does not suffice to express complex mathematical constraints such as are expressed in formulae such as CnH2n. 

This should also allow expressions in terms of absences of atoms of certain types (see Despoina's tech report). 



\section*{Discussion}

%% First we discuss the findings of the above evaluations for structural classification in chemistry.  Which approaches are best suited to which class types?  What are the strengths and weaknesses of chemoinformatics approaches and chemical ontology approaches?  What is the motivation for doing chemical ontology? 


%  Human knowledge and black box classification.
In contrast to the above methods of automatic hierarchy construction, chemical ontology consists in the specification of a hierarchy ``from the top down'', in the sense that the features of chemical classes are specified, and their members are assigned based on these features.  Chemical classes and the relationships between them may be specified to a greater or lesser level of detail, not restricted by what the algorithms for detecting similarity or substructures are able to detect. Creating such a hierarchy allows for the explicit representation of knowledge in the domain, knowledge which corresponds to the content of textbook chemistry and which can then be directly correlated with research reports in the literature as well as large-scale databases of chemical compounds. % Cite Expert-Systems? Dendral?

The explicit representation of knowledge in this fashion allows for the classification of edge cases (unusual classes) and cases which cannot be treated within the constraints of the available algorithmic tools. Statistical (machine-learning) approaches rely on the underlying quantification of features in the molecules -- and features which are not common are less likely (vanishingly unlikely?) to be represented in resulting trained models. Similarity comparison (used in pairwise similarity-based hierarchy construction) are also vulnerable to the specification of features to be used in the quantification of similarity. Also, many of the features used are path-based, that is, they traverse combinatorially exhaustive paths through the molecule \textit{up to a certain length}.  It is difficult to capture overall features of the molecule with path-based approaches.  However, some overall features of molecules, such as count of rings, are often added in to the features used in such classifications. Substructure detection is similarly unable to account for overall features of molecules. 

Examples of edge classes which appear difficult to deal with in the chemoinformatics approaches are thus:
\begin{enumerate}
	\item cyclic peptide (because the cycle in question is not an arbitrary attached ring, but a cycle of chained peptide links)
\end{enumerate}


Machine classification works best where the features are small and they can be learned from a large collection of instances.  Ontology-based classification using logical definitions gives a flexibility in defining features, even very large ones, or ones that span over a small number of examples but are nevertheless important.  The most important thing is that the eventual classification (howsoever arrived at) is PROVABLY CORRECT, so that it does not lead to incorrect inferences and can be used to support science in novel ways (more about that in the concluding remarks on applications)


Reasons that ontology and machine classification as currently used in chemoinformatics are distinct, mutually non-reducible exercises:  

1) NAMING of mid-level groupings. Where chemists have, for whatsoever historical tradition, already a name in use for a particular class of chemical entities, machine learned groupings may not discover quite the exact grouping that is intended to be referred to by that name. This leads to the situation where it is not possible, for example, to group together all the literature describing that category of chemicals, despite the fact that chemists think and communicate regularly in terms of such categories.  This can be compared to the scenario in chemistry education, where relevant groupings of chemical entities are often taught in chapter-specific units. %Need an example here!  Compare the output of a hierarchical clustering algorithm based on skeletons, to the content of a chemistry text book.  

2) Association of functions to mid-level groupings.  Grouping chemical structures by shared functions is one of the essential elements in chemoinformatics approaches.  If it is possible to group together all molecules which act against the same receptor, it is then possible to train predictive models based on this information.   %Cite something Willighagen-like here
Research in the sciences often examines groupings of chemical entities which exhibit shared behaviour in order to understand more about the mechanisms underlying that behaviour, as well. %Can cite that talk on the neural mechanisms of odor perception here, they used a principle component analysis on the structures of molecules which were known to be odorants. 
Having to extract the grouping that you are interested in manually from the database by doing a literature analysis yourself in every case is a labour-intensive task, and it is one that should be centralised so as to free up the resources of researchers for focusing on their primary research. 
Very importantly, this sort of information needs to be hierarchically organised, so that it is not repetitively described, and so that it can be grouped and clustered at different levels of aggregation depending on the needs of the individual researcher.  For some research purposes, one may be interested in the classification of all molecules which are odorants. For other purposes, one may be interested in only those which smell sweet or smell bitter. 

3) Those classes which represent structural features which are beyond the reach of our current chemoinformatics tools for automatically clustering and classifying our chemical structures. For example, fullerenes, cyclic peptides, and other examples. 

For these reasons, chemical ontology has a valid place alongside the more automatic methods for chemical classification.  However, this presents a challenge for tooling and for algorithm research, in that the logic-based ontology tools and algorithms need to work alongside chemoinformatics tools and algorithms. 


Harnessing the strengths of both approaches in complementary systems \ldots

%% What are the challenges in harnessing chemoinformatics from within a logical framework?  What are the needs on integrative approaches?

%% What will a hybrid system look like and what is it good for? 


Pre-computing and asserting all parts as a workaround to the problem with directly modelling structured objects: 
-- Explosion of asserted parts, and the parthood / OWL axioms challenge (if you precompute and assert all the parts). Not only parts, but also properties. Hence the need for Despoina's work so that graph structures can be exposed to the ontology directly. 
-- Substructure search and similarity search already well developed in chemoinformatics 





\section*{Conclusions}

summary of ongoing research and future directions

Artificial intelligence of the future and the need for hybrid systems in addition to the current and ongoing developments 

%%TODO: Check the papers which cite ChEBI for examples of how chemical ontologies (and especially ChEBI) are used!

-- describe applications of chemical ontology in text mining, data integration, automated reasoning, visualisation and biology (pathways?)

exciting applications of the future too: nanotechnology and biotechnology 



\section*{Methods}

\subsection*{Representing the ChEBI ontology in OWL}

-- OWL API
-- ChEBI ontology data model and the ChEBI maintenance suite, in particular the tooling for chemical structures
-- Conversion of relationships into all-some restrictions
-- InChI as annotations

\subsection*{Representing chemicals as description graphs}
-- Model for molecules (MOLfile to DG)
-- Software involved
-- How we create description graphs
-- Model for DGs at the class level (not the fully specified;)
-- Relationship to SMARTS, R-Groups

\bigskip


%%%%%%%%%%%%%%%%%%%%%%%%%%%%%%%%
\section*{Author's contributions}
    Text for this section \ldots

    

%%%%%%%%%%%%%%%%%%%%%%%%%%%
\section*{Acknowledgements}
  \ifthenelse{\boolean{publ}}{\small}{}
  Text for this section \ldots
 
%%%%%%%%%%%%%%%%%%%%%%%%%%%%%%%%%%%%%%%%%%%%%%%%%%%%%%%%%%%%%
%%                  The Bibliography                       %%
%%                                                         %%              
%%  Bmc_article.bst  will be used to                       %%
%%  create a .BBL file for submission, which includes      %%
%%  XML structured for BMC.                                %%
%%  After submission of the .TEX file,                     %%
%%  you will be prompted to submit your .BBL file.         %%
%%                                                         %%
%%                                                         %%
%%  Note that the displayed Bibliography will not          %% 
%%  necessarily be rendered by Latex exactly as specified  %%
%%  in the online Instructions for Authors.                %% 
%%                                                         %%
%%%%%%%%%%%%%%%%%%%%%%%%%%%%%%%%%%%%%%%%%%%%%%%%%%%%%%%%%%%%%

\newpage
{\ifthenelse{\boolean{publ}}{\footnotesize}{\small}
 \bibliographystyle{bmc_article}  % Style BST file
  \bibliography{strucontjcheminf} }     % Bibliography file (usually '*.bib' ) 

%%%%%%%%%%%

\ifthenelse{\boolean{publ}}{\end{multicols}}{}

%%%%%%%%%%%%%%%%%%%%%%%%%%%%%%%%%%%
%%                               %%
%% Figures                       %%
%%                               %%
%% NB: this is for captions and  %%
%% Titles. All graphics must be  %%
%% submitted separately and NOT  %%
%% included in the Tex document  %%
%%                               %%
%%%%%%%%%%%%%%%%%%%%%%%%%%%%%%%%%%%

%%
%% Do not use \listoffigures as most will included as separate files

\section*{Figures}
  \subsection*{Figure 1 - Sample figure title}
      A short description of the figure content
      should go here.

  \subsection*{Figure 2 - Sample figure title}
      Figure legend text.



%%%%%%%%%%%%%%%%%%%%%%%%%%%%%%%%%%%
%%                               %%
%% Tables                        %%
%%                               %%
%%%%%%%%%%%%%%%%%%%%%%%%%%%%%%%%%%%

%% Use of \listoftables is discouraged.
%%
\section*{Tables}
  \subsection*{Table 1 - Sample table title}
    Here is an example of a \emph{small} table in \LaTeX\ using  
    \verb|\tabular{...}|. This is where the description of the table 
    should go. \par \mbox{}
    \par
    \mbox{
      \begin{tabular}{|c|c|c|}
        \hline \multicolumn{3}{|c|}{My Table}\\ \hline
        A1 & B2  & C3 \\ \hline
        A2 & ... & .. \\ \hline
        A3 & ..  & .  \\ \hline
      \end{tabular}
      }



%%%%%%%%%%%%%%%%%%%%%%%%%%%%%%%%%%%
%%                               %%
%% Additional Files              %%
%%                               %%
%%%%%%%%%%%%%%%%%%%%%%%%%%%%%%%%%%%

\section*{Additional Files}
  \subsection*{Additional file 1 --- Sample additional file title}
    Additional file descriptions text (including details of how to
    view the file, if it is in a non-standard format or the file extension).  This might
    refer to a multi-page table or a figure.

  \subsection*{Additional file 2 --- Sample additional file title}
    Additional file descriptions text.


\end{bmcformat}
\end{document}







